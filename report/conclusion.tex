\section{Conclusion}
\label{sec:conclusion}

Dans ce rapport, nous avons évalué l'impact du renommage d'entités sur les valeurs des métriques de procédés logiciels. Nous avons effectué une étude empirique sur cinq projets open-source connus et matures. Nous avons observé que les périodes initiales des projets sont plus enclines à contenir du renommage que les autres périodes. Plus important, nous avons constaté que d'autres périodes peuvent contenir une quantité importante de renommage, en particulier celles correspondantes à la mise au point de releases majeures. Enfin, nous avons observé que le renommage pouvait biaiser considérablement les valeurs des métriques de procédés. 

Par conséquent, nous avons mis en évidence que les chercheurs et développeurs devraient être prudents lors du calcul des métriques de procédés pour éviter de biaiser leurs conclusions. Nous avons pu leur faire quelques recommandations, comme d’éviter le calcul des métriques de procédés lors des périodes initiales et d’utiliser un algorithme de détection de renommage lors du calcul des métriques de procédés sur les autres périodes.\\

Suite à cette étude, il est prévu d'évaluer la précision des algorithmes existants de détection de renommage et d'évaluer l'impact de la fusion de code (code merging) sur les métriques de procédés.
