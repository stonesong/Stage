\section{Conclusion}
\label{sec:conclusion}

Dans cet article, nous avons évalué l'impact du renommage d'entités sur les valeurs des métriques de procédés logicielles. Nous avons effectué une étude empirique sur cinq projets open-source connus et matures. Nous avons observé que les périodes initiales des projets sont plus enclines à contenir du renommage que les autres périodes. Plus important, nous avons constaté que d'autres périodes peuvent contenir une quantité importante de renommage, en particulier celles correspondantes à la mise au point de releases majeures. Enfin, nous avons observé que le renommage pouvait biaiser considérablement les valeurs des métriques de procédés. Par conséquent, les chercheurs et développeurs devraient être prudent lors du calcul des métriques de procédés. Nous recommandons d'éviter le calcul des métriques de procédés lors des périodes initiales. Pour finir, nous recommandons fortement d'utiliser un algorithme de détection de renommage lors du calcul des métriques de procédés sur d'autres périodes tels qu'il pourrait en biaiser fortement le résultat.\\

A l'avenir, nous prévoyons d'évaluer la précision des algorithmes existants de détection de renommage. Nous prévoyons également d'évaluer l'impact de la fusion de code (code merging) sur les métriques de procédés.
