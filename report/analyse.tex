\section{Analyse des études passés}
\label{sec:analyse}

In this section, we proceed to an analysis of the past studies that used change metrics to predict defects. We evaluate if the values of changes metrics could be biased by analyzing how they collected their data. Finally, we give some guidelines to help researchers and practitioners to avoid the impact that artifact renaming has on change metrics.

\subsection{Analysis of Past Studies}

Firstly, it is important to remark that as we have shown in \secref{results}, periods containing a high amount of renaming are rare. Therefore, most of the past studies should be not affected by this phenomenon. Additionally, even in the case of using periods having a significant amount of renaming, the results of such studies could also be improved, because change metrics would probably have been underestimated. Nevertheless, several past studies can be impacted by renaming, as we will point out in the remainder of this section. Quantifying such effect on these past studies is out of the scope of this paper, but we provide guidelines for future studies in \secref{guidelines}.

In this analysis of past studies, we include only the 26 studies referenced in~\cite{radjenovic_software_2013} that use the CC, NoD or NoC metrics. However several other studies referenced in~\cite{radjenovic_software_2013} use slightly different change metrics and could also be impacted on renaming.

Nous nous sommes donc intéressés aux études passées qui pouvaient traiter les trois métriques de procédés cités ci-dessus dans la prédiction de bugs, et vérifié si ces études avaient considéré le renommage de fichiers. L'article ~\cite{radjenovic_software_2013} de Rajenovi et al référence $26$ études sur ce sujet.\\

$15$ de ces études analyses des projets industriels, ~\cite{arisholm_systematic_2010,graves_predicting_2000,khoshgoftaar_using_2000,layman_iterative_2008,munson_code_1998,nagappan_use_2005,nagappan_influence_2008,nagappan_using_2007,nagappan_using_2006,nagappan_change_2010,nikora_building_2006,ostrand_programmer-based_2010,weyuker_too_2008,weyuker_using_2007,yuan_application_2000}. Aucune de ces études ne parle de renommage, mais le manque d'information récoltées sur les VCS utilisés et sur le projet en lui-même ne nous permet pas de savoir si le renommage pouvait avoir un impact sur ces projets. Néanmoins, l'article de Kim et al ~\cite{kim_field_2012} explique que les développeurs dans son étude effectuent des opérations de refactoring, dont du renommage, sans utiliser les outils du VCS appropriés. Ainsi, ces études pourraient être impactées par le renommage en fonction des outils utilisés et des habitudes de développement.\\

$11$ études analysent des logiciels open-source \cite{dambros_relationship_2009,bacchelli_are_2010,caglayan_merits_2009,dambros_evaluating_2012,dambros_evaluating_2012,dambros_extensive_2010,illes-seifert_exploring_2010,li_finding_2005,matsumoto_analysis_2010,moser_analysis_2008,moser_comparative_2008,schroter_if_2006}. Les VCS utilisés dans ces études sont CVS ou Subversion. CVS ne gère pas le renommage et Subversion uniquement de manière manuelle ce qui est dangereux comme expliqué dans l'article ~\cite{lavoie_inferring_2012,steidl_incremental_2014}. Seulement deux de ces études ~\cite{moser_analysis_2008,moser_comparative_2008} parlent de renommage dans leur set de données ou dans les ''Threats to validiy''. Pour réduire le risque d'erreur dans leurs expérimentations, ces deux études ont supprimé systématiquement tous les fichiers ajoutés ou supprimés durant les périodes analysées. C'est un bon moyen d'éviter de calculer des métriques de procédés biaisés, mais cela implique aussi de supprimer inutilement du jeu de données un nombre significatif de fichiers.\\

\subsection{Guidelines}
\label{sec:guidelines}

According to the results of our two experiments, we deduce several simple guidelines to compute change metrics. Firstly, we recommend to avoid computing such metrics during initial periods of projects at all costs. Indeed, these periods usually contain a significant amount of renaming. As we have seen, both major and minor periods can contain a significant amount of renaming, although major release seems more prone to renaming. In any case, we recommend to systematically use a renaming detection algorithm, to avoid picking up the wrong period. Git provides a dedicated algorithm that seems to have a good precision, but an unknown recall. Therefore using projects managed by Git seems the easier way to lower the threat of renaming. More advanced renaming detection algorithms are also described in the literature:~\cite{antoniol_automatic_2004,lavoie_inferring_2012,steidl_incremental_2014}. They have been empirically validated so they might perform better than Git's algorithm, and are the only choices if the chosen corpus contains project that are not managed by Git. Finally, for change metrics computed at finer level of granularity than files, we recommend the use of origin analysis algorithms such as~\cite{wu_aura:_2010}. These algorithms usually work at the granularity of the functions. Finally, as artifact renaming can be a significant threat, we recommend to systematically indicate how it was dealt with in future studies.
