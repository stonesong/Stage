\section{Un ensemble de projet}
\label{sec:ensemble_projet}
Nous avons donc dû sélectionner un ensemble de projets sur lesquels effectuer nos expérimentations qui respectent le modèle défini. Des projets open-source, conséquents et connues de la communauté MSR. Nous avons un ensemble de projets utilisés par l'équipe de Génie Logiciel au LaBRI qui respectent le modèle avec des branches de maintenances identifiées. Les $5$ projets qui sont donnés \tabref{projects} nous fournissent un corpus pour notre prochaine expérience avec différents langages de programmation, un nombre de lignes de code ainsi qu'un nombre de développeurs dans la moyenne jusqu'à élever par rapport aux projets open source utilisés par la communauté. Les $5$ projets sont gérés sur Git afin de profiter de la détection automatique des renommages (section). \\
De plus, il faut noter que nous avons choisi d'exclure tous les fichiers qui ne sont pas du code source du corpus étant donné que les métriques de procédés sont habituellement uniquement calculées sur ces fichiers. \\

\begin{table*}[t]
\centering
\begin{tabular}{rcccc}
\toprule
Project & Main language & Size (LoC) & Number of developers & URL\\
\midrule
Jenkins & Java & 200851 & 454 & \url{github.com/jenkinsci/jenkins} \\
JQuery & JavaScript & 41656 & 223 & \url{github.com/jquery/jquery} \\
PHPUnit & PHP & 21799 & 152 & \url{github.com/sebastianbergmann/phpunit}\\
Pyramid & Python & 38726 & 205 & \url{github.com/Pylons/pyramid} \\
Rails & Ruby & 181002 & 2767 & \url{github.com/rails/rails}\\
\bottomrule
\end{tabular}
\caption{Our corpus of software projects.}
\label{tab:projects}
\end{table*}
