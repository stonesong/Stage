\section{Introduction}
\label{sec:intro}

L'apparition des premiers dépots logiciels en libre accès dans les années 90, des stockages centralisés de donnés, a rendu possible de nombreux travaux de recherche sur l'évolution logicielle. Plus particulièrement avec les dépôts de code source gérés par des outils de contrôle de versions (Version Control System, VCS) tels que  SVN (2000), Mercurial (2005) ou encore Git (2005) qui contiennent l'historique de construction d'un logiciel.\\  

C'est principalement dans le domaine du ``Reverse Engineering'', qui permet decomprendre les choix des développeurs lors de la création d'un logiciel, qu'il existe des études qui se basent sur l'analyse de ces historiques. Elles entrent dans le cadre des études ``MSR'' (Minning Software Repository).

De même, la prédiction de bugs, un des défis connus du Génie Logiciel dont le but est de prédire le nombre de bugs et leur localisation dans la prochaine version d'un logiciel, utilise des informations contenues dans l'historique d'un projet. Le principe repose sur les métriques de procédés comme prédicateurs de bugs.

Les métriques de procédés se concentrent sur l'évolution d'un logiciel et mesurent les modifications subies par les entités d'un code source durant leur cycle de vie. L'hypothèse principale étant que la manière dont les entités du code ont changé a un impact majeur sur leur qualité et donc sur les bugs qu'elles peuvent générer. Il est donc primordial que les mesures des métriques de procédés représentent au plus proche la réalité des changements.\\

Or au cours de son histoire, un fichier peut être renommé et/ou déplacé dans un autre dossier du projet.

Théoriquement, si le renommage d'un fichier à un moment donné de son histoire n'est pas pris en compte, le calcul d'une métrique de procédé sur ce fichier sera faussé. En effet, dans le cas où un fichier est identifié par son nom, les informations récoltées avant le renommage seront perdues. Par ailleurs, il est de notoriété commune que le refactoring, modifications architecturales qui permettent d'améliorer le code source, dont le renommage de fichiers, est très présent dans le développement des logiciels à succès d'aujourd'hui. En pratique, nous ne disposons pas de chiffres pour en connaitre l'ampleur.\\

L’objet de nos travaux est donc d’étudier les pistes qui peuvent nous conduire à mettre en évidence les renommages, les récupérer et effectuer certaines statistiques.
Dans un premier temps, nous présentons état de l'art sur les méthodes utilisées pour détecter le refactoring, les logiciels qui ont été étudiés, les métriques de procédés ainsi que les VCS. 

Compte tenu de l'état de l'art nous exposons la problématique à résoudre, qui nous a conduit à redéfinir le renommage et les niveaux de granularités. Puis, réaliser une analyse manuelle sur un premier projet afin de récupérer les renommages réels. Ces travaux se sont poursuivis par la définition d'un modèle, le choix d'un ensemble de projets cohérant pour faire nos propres expérimentations et la creation d'un outil pour récupérer les renommages.

 Enfin, nous décrivons comment calculer certaines métriques de procédés et mesurons l'impact du renommage.\\
 Les résultats de nos expérimentations nous ont amenés à proposer un article pour la conférence ICSME 2014.\\

