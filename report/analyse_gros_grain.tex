\section{Analyse à gros grain}
\label{sec:analyse_gros_grain}

\subsection{Première experience}

Plus précisément voici le déroulement général de notre outil:

Avant tout nous avons besoin de la liste des tags de release trié dans l'ordre chronologique. Cette partie ne peut être automatisé en raison des conventions de nom de tags différentes entre chaque projets. Par exemple: 
\begin{itemize}
\item PHPunit: \texttt{3.5.0, 3.6.0} etc.
\item Pyramid: \texttt{1.0, 1.1} etc.
\item Jenkins: \texttt{jenkins-1\_400, jenkins-1\_410} etc.
\item Rails: \texttt{v2.0.0, v2.1.0} etc.
\end{itemize}
\medskip
Enfin voici le déroulement du processus :

Nous parcourons les branches distantes (\texttt{remotes}) du dépôt Git du projet. Toute les branches sont considérés comme branche de maintenance, sauf la branche \texttt{master}. La plus part du temps, les branches spécifiques de maintenance contiennent ``-stable'' dans leur nom, mais si une branche est listée ici, cela veut dire que la tête de branche, c'est à dire sa dernière version, est séparée et n'est pas joignable de la branche \texttt{master}. C'est donc une branche de maintenance depuis son dernier \texttt{merge} avec master, c'est à dire depuis leur dernière fusion. Il est laissé à la discrétion de chacun d'éliminer les branches qui ne sont pas spécifiquement de maintenance au préalable, mais dans nos expériences, ces branches n'ont pas faussé notre étude. En général les branches présentes sur le dépôt qui ne sont ni \texttt{master} ni de maintenance ne contennaient peux ou pas de \texttt{commits}.

La branche \texttt{origin/master} contient la partie initiale (init) et la partie de développement (dev).\\  
Le travail est donc réalisé sur: 
\begin{itemize}
\item Les branches de maintenance
\item La branche \texttt{origin/master}
\end{itemize}
\medskip
Si la branche est \texttt{origin/master}, le travail est divisé encore dans les périodes:
\begin{itemize}
\item Du premier commit(inclue) jusqu'au premier tag de release.
\item Du premier jusqu'au dernier tag (une release après l'autre forme une période).*
\end{itemize}
\medskip

(* nous ne considérons pas la période du dernier tag jusqu'à la tête de la branche master, car la release n'étant pas terminé, le nombre de commit ne sera pas de la taille d'une release normale et les chiffres obtenus sur cette partie ne seront donc pas pertinants)

Ainsi, pour chacune de nos trois conditions, maintenance(maint), période initiale(init) et releases(dev) qui formes l'ensemble des périodes que nous analysons, nous allons générer le \texttt{log}. Le \texttt{log} donné par la commande \texttt{git log} contient l'ensemble des \texttt{commits} dans nos périodes avec toute les informations nécessaires pour chacun d'eux.

La majeure partie du travail réalisé se situe sur ces \texttt{log}. Nous avons choisis de travailler sur un \texttt{log} général plutôt qu'en récupérant l'historique de chaque fichier (avec l'option \texttt{``--follow''}) l'un après l'autre pour des raisons de performances. Particulièrement sur les gros projets tels que Rails ou Jenkins, cette dernière technique n'est presque pas réalisable. De plus elle n'est réalisable que sur les fichiers présent à la tête de la branche et pas à partir d'une version antérieure, ce qui impliquerais un travail supplémentaire pour configurer le dépôt à chaque release sur la branche \texttt{master}.\\

Nous listons les fichiers existants à la fin de chaque période. Dans notre expérience nous écartons les fichiers qui ne sont pas du code source. 
Enfin l'algorithme principal, qui parcours des informations contenus par les \texttt{log} dans l'ordre chronologique, va nous permettre de suivre les renommages et les modifications et ainsi reconstruire l'histoire des fichiers. Un renommage typique ressemble à \texttt{``rename bob/\{henry $\Rightarrow$ josef\}/george.py (86\%)''}. Si \texttt{''bob/henry/george.py''} à été enregistré au préalable par notre algorithme nous suivrons maintenant \texttt{''bob/josef/george.py''} à la place. Il y a de nombreux cas particuliers à prendre en compte et de nombreuses applications possible par rapport aux informations récupérés ainsi. On peut choisir de ne considérer que le dernier nom d'un fichier par exemple ou stocker tout ses noms précédents. on élimine finalement tout les fichiers qui ne sont pas dans notre liste de établie au préalable.\\

(TODO partie vérif?)\\

\subsection{Deuxième expérience}

\subsubsection{Calcul des métriques}



\subsubsection{Calcul de corrélation}


