\section{Introduction}
\label{sec:intro}

L'accès aux dépôts logiciels a rendu possible de nombreux travaux de recherche sur l'évolution logicielle. Plus particulièrement, les dépôts de code source gérés par des outils de contrôle de versions (Version Control System, VCS, comme SVN, Mercurial ou encore Git) contiennent l'historique de construction d'un logiciel. Des études se basent sur l'analyse de ces historiques. Principalement dans le ``Reverse Engineering'', la compréhension des choix des développeurs lors de la création d'un logiciel, ou encore la prédiction de bugs un des défis connus du Génie Logiciel, dont le but est de prédire le nombre de bugs et leur localisations dans la prochaine version d'un logiciel. Cette étude se base sur les métriques de procédés comme prédicteurs de bugs. Les métriques de procédés se concentrent sur l'évolution d'un logiciel et mesures les modifications subies par les entités d'un code source durant leur cycle de vie. L'hypothèse principale étant que la manière dont les entités du code ont changés a un impact majeur sur la qualité de leur prédiction de bugs.\\
Or au cours de son histoire, un fichier peut être renommé et/ou déplacé dans un autre dossier du projet.\\
Théoriquement, si le renommage d'un fichier à un moment donné de son histoire n'est pas pris en compte, le calcul d'une métrique de procédé sur ce fichier sera faussé. En effet, si on identifie le fichier par son nom, on perdra les informations récoltés avant le renommage. Par ailleurs on peut penser que le refactoring, dont le renommage de fichiers, est très present dans le dévelopement des logiciels à succès d'aujourd'hui. En pratique, nous n'avons pas de chiffres pour le montrer.\\ 
Dans un premier temps nous effectuerons une étude de l'existant sur les méthodes utilisés pour détecter le refactoring, les logiciels qui ont étés étudiés, les métriques de procédés ainsi que les VCS. Puis nous choisirons un ensemble de projets cohérant pour faire nos propres expérimentations, nous définirons un niveau de granularité et nous ferons une analyse manuelle des projets choisis pour récupérer les renommages réels. Par la suite nous définirons un modèle et nous utiliserons un outils pour récupérer les renommages. Enfin nous définirons comment calculer certaines métriques de procédés et mesurerons l'impact du renommage. Les résultats de nos expérimentations améneront à une publication dans la conférence ICSME 2014.\\

