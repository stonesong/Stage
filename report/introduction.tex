\section{Introduction}
\label{sec:intro}

(TODO: déf dépôt logiciel, date etc) L'accès aux dépôts logiciels a rendu possible de nombreux travaux de recherche sur l'évolution logicielle. Plus particulièrement, les dépôts de code source gérés par des outils de contrôle de versions (Version Control System, VCS, comme SVN, Mercurial ou encore Git) qui contiennent l'historique de construction d'un logiciel. (TODO: mot coupé) C'est principalement dans le domaine du ``Reverse Engineering'', qui est la compréhension des choix des développeurs lors de la création d'un logiciel, qu'il existe des études qui se basent sur l'analyse de ces historiques. Elles entrent dans le cadre des études ``MSR'' (Minning Software Repository).\\
De même, la prédiction de bugs, un des défis connus du Génie Logiciel dont le but est de prédire le nombre de bugs et leur localisation dans la prochaine version d'un logiciel, utilise des informations contenues dans l'historique d'un projet. Le principe repose sur les métriques de procédés comme prédicateurs de bugs.\\
Les métriques de procédés se concentrent sur l'évolution d'un logiciel et mesurent les modifications subies par les entités d'un code source durant leur cycle de vie. L'hypothèse principale étant que la manière dont les entités du code ont changé a un impact majeur sur leur qualité et donc sur les bugs qu'ils peuvent générer. Il est donc primordial que les mesures des métriques de procédés représentent au plus proche la réalité des changements.\\
Or au cours de son histoire, un fichier peut être renommé et/ou déplacé dans un autre dossier du projet.\\
Théoriquement, si le renommage d'un fichier à un moment donné de son histoire n'est pas pris en compte, le calcul d'une métrique de procédé sur ce fichier sera faussé. En effet, dans le cas ou un fichier est identifié par son nom, les informations récoltées avant le renommage seront perdus. Par ailleurs, il est de notoriété commune que le refactoring, dont le renommage de fichiers, est très présent dans le développement des logiciels à succès d'aujourd'hui. En pratique, nous n'avons pas de chiffres pour le montrer.\\ 
Dans un premier temps, nous présentons état de l'art sur les méthodes utilisées pour détecter le refactoring, les logiciels qui ont été étudiés, les métriques de procédés ainsi que les VCS. Puis nous avons choisi un ensemble de projets cohérent pour faire nos propres expérimentations, défini un niveau de granularité et nous faisons une analyse manuelle des projets choisis pour récupérer les renommages réels. Par la suite, nous a conduit à définir un modèle et à créer un outil pour récupérer les renommages. Enfin, nous décrivons comment calculer certaines métriques de procédés et mesurons l'impact du renommage. Les résultats de nos expérimentations nous ont amener à proposer un article pour la conférence ICSME 2014.\\

