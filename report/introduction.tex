\section{Introduction}
\label{sec:intro}

L'apparition des premiers dépôts logiciels en libre accès dans les années 90~\cite{list_vcs} a rendu possible de nombreux travaux de recherche sur l'évolution logicielle. Plus particulièrement avec les dépôts de code source gérés par des outils de contrôle de versions (Version Control System, VCS) tels que  SVN (2000), Mercurial (2005) ou encore Git (2005) qui contiennent l'historique de construction d'un logiciel.\\  

C'est principalement dans le domaine de la maintenance et évolution logicielle, qui permet de comprendre les choix des développeurs lors de la création d'un logiciel, qu'il existe des études se basant sur l'analyse de ces historiques. Elles entrent dans le cadre des études ``MSR'' (Mining Software Repositories)~\cite{herzig-makingsoftware-2010} qui consistent à analyser les données rendues disponibles par les dépôts logiciels.

On trouve dans les applications de ce domaine, un défi connu du Génie Logiciel, la prédiction de bugs. Son but est de prédire le nombre de bugs et leurs localisations. Plus précisément, se sont des calculs de métriques sur les entités du code source qui composent un projet, qui lui permettront de prédire quelle entité contiendra combien de bugs. Une métrique est calculée sur une entité à partir de ses composantes techniques ou fonctionnelles. Elle peut être calculée en regardant l'évolution de l'entité dans le cycle de vie du logiciel~\cite{dambros_relationship_2009,caglayan_merits_2009}. Dans ce cas, l'analyse s'appuiera sur les informations contenues dans l'historique du projet.

Un grand nombre d'études, dans le contexte de la prédiction de bugs, tente alors de déterminer quelles sont les meilleures métriques qui serviront de prédicteurs de bugs~\cite{fenton_critique_1999}.

Nagappan \emph{et al}~\cite{nagappan_mining_2006} montrent que ni les métriques telles que le nombre de lignes de code (LoC, Lines of Code) ni les métriques orientées objets proposés par Chidamber \emph{et al}~\cite{chidamber_metrics_1994} telles que la profondeur de l'arbre d'héritage (DIT, Depth of Inheritance Tree) ne peuvent être utilisés dans tous les logiciels. Depuis ces travaux, beaucoup d'études plus récentes montrent que les métriques de procédés (software process metrics) donnent de bien meilleurs résultats ~\cite{nagappan_use_2005,weyuker_too_2008,bird_dont_2011,giger_can_2012}.\\

Les métriques de procédés se concentrent sur l'évolution d'un logiciel et mesurent les modifications subies par les entités d'un code source durant leur cycle de vie. L'hypothèse principale est que la façon dont les entités du code ont changé, a un impact majeur sur leur qualité et donc sur les bugs qu'elles peuvent contenir. Il est donc primordial que les valeurs des métriques de procédés représentent au mieux la réalité des changements.

Radjenovic \emph{et al}~\cite{radjenovic_software_2013} identifient les trois métriques de procédés les plus utilisés. Le nombre de développeurs ~\cite{weyuker_too_2008} (Number of Developers, NoD), le nombre de modifications ~\cite{graves_predicting_2000} (Number of Changes, NoC) et le Code Churn ~\cite{munson_code_1998} (CC). NoD compte le nombre de développeurs qui ont contribués à une entité. NoC compte le nombre de changements qu'a subi une entité. CC compte le nombre de lignes de code qui ont été ajoutées ou supprimées à une entité.\\

Pour un logiciel donné, calculer les métriques de procédés consiste à observer tous les changements subis par chaque entité qu'il contient. Dans ce but, l'utilisation d'un gestionnaire de version (VCS) est indispensable car il permet de suivre les changements effectués par tous les développeurs sur toutes les entités.\\ 

Or, au cours de son histoire, une entité du code source telle qu'un fichier, peut être renommée et/ou déplacée dans un autre dossier du projet. Ces actions sont peu ou pas prises en compte par les VCS, ce qui rend le calcul des métriques plus délicat et sujet aux erreurs.\\

Théoriquement, si le renommage d'une entité à un moment donné de son histoire n'est pas pris en compte, le calcul d'une métrique de procédé sur ce fichier sera faussé. En effet, dans le cas où un fichier est identifié par son nom, les informations disponibles avant le renommage seront perdues. Par ailleurs, il est de notoriété commune que les \textit{refactorings}, modifications architecturales (dont le renommage d'entités) qui permettent d'améliorer ou réformer le code source, sont très utilisées au cours de la construction des logiciels. En pratique, nous ne connaissons pas la quantité de renommage ni son impact sur les métriques de procédés.\\

L'objet de nos travaux est donc d’étudier le phénomène du renommage et son impact sur les métriques de procédés. Dans ce but, nous apportons les contributions suivantes:\\

Tout d'abord, nous effectuons une étude empirique sur cinq projets avec l'intention de donner un aperçu de la quantité d'entités renommés dans les logiciels. Nous présentons alors cette étude empirique approfondie du renommage d'entité sur cinq projets \textit{open-source} connus et matures. Nous donnons également des informations détaillées sur la quantité de renommage.

Ensuite, nous étudions l'impact du renommage d'entités en calculant les trois métriques de procédés les plus utilisés, sur les projets de notre corpus avec et sans considération du renommage d'entités. Nos résultats indiquent que la quantité de renommage qui apparait dans les projets peut être importante. Nous avons observé jusqu'à $99\%$ d'entités renommées dans un projet. Nous avons aussi observé que le renommage peut affecter les valeurs des métriques de procédés de manière significative, et par conséquent qu'il peut être un obstacle sérieux à la validité des études utilisant ces métriques. Nous apportons ici une analyse de l'impact du renommage sur les métriques de procédés.

Basée sur nos observations, nous proposons une analyse de l'impact possible du renommage sur les études antérieures qui traitent de la prédiction de bugs. 

Enfin, nous proposons quelques recommandations essentielles qui aideront les chercheurs et développeurs à mieux calculer les métriques de procédés et à limiter l'impact du renommage sur les calculs de métriques de procédés.\\

Les résultats de nos expérimentations nous ont amenés à proposer un article pour la conférence internationale ICSME 2014~\cite{icsme_2014}.


% L’objet de nos travaux est donc d’étudier le phénomène du renommage et son impact sur les métriques de procédés. Nous présentons une étude empirique approfondie sur cinq projets open-source connus et matures avec l'intention de donner une idée du montant d'entités renommés dans les logiciels. De plus, nous étudions l'impact du renommage d'entités en calculant les trois métriques de procédés les plus utilisés sur les projets de notre corpus avec et sans considération du renommage d'entités. Nos résultats indiquent que la quantité de renommage qui apparait dans les projets peut être importante. Nous avons observé jusqu'à $99\%$ d'entités renommés dans un projet. Nous avons aussi observé qu'il peut affecter les valeurs des métriques de procédés de manière significative, et par conséquent qu'il peut être une menace sérieuse d'études utilisant ces métriques. Basée sur nos observations, nous proposons une brève analyse de l'impact possible du renommage sur les études antérieures qui traitent de la prédiction de bugs. Nous proposons enfin de simples lignes de conduite qui aideront les chercheurs et développeurs à mieux calculer les métriques de procédés.\\

% Pour résumer les contributions que nous apportons:
% \begin{itemize}
% \item Une étude empirique du renommage d'entités sur cinq projets open-source populaires.    
% \item Des informations détaillées sur la quantité de renommage.
% \item Une analyse de l'impact du renommage sur les métriques de procédés.
% \item Une analyse des études antérieures qui utilisent les métriques de procédés pour la prédiction de bugs, afin d'évaluer si elles peuvent être biaisées par le renommage. 
% \item Quelques lignes de conduite essentielles afin de limiter les risques lors des calculs des métriques de procédés.\\
% \end{itemize}


