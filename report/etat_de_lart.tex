\section{Etat de l'art}
\label{sec:etat_de_lart}

\subsection{Evolution logiciel et refactoring}
Pour commencer à comprendre l'évolution des logiciels et la place du refactoring, nous avons d'abord cherché les articles qui mentionnaient le refactoring. Puis nous avons cherché des études de MSR qui pourraient mettre en avant des chiffres à propos du refactoring, par exemple un pourcentage d'opérations de refactoring ou de commits contenant du refactoring. On peut régulièrement lire en introduction d'articles dans le domaine, des propos sur l'importance du refactoring, qui inclut le renommage, et sur l'intérêt des techniques de compréhension de l’évolution des architectures et structures des logiciels. Il est de notoriété commune que les logiciels à succès sont généralement amenés à évoluer dans le temps et à se restructurer après la découverte de bugs, l’ajout de fonctionnalités ou l’adaptation à l’environnement dans lequel ils évoluent. Le maintien d’un tel logiciel passe par la compréhension des choix d’architecture pris par le passé, c'est-à-dire par son histoire. ~\cite{tu_integrated_2002,godfrey_tracking_2002,kim_field_2012}.

 Néanmoins nous n'avons pas obtenu de chiffres précis sur le nombre de renommage, le nombre d'opérations de refactoring ou autres. Uniquement dans l'étude de Kim et al, qui nous donne un pourcentage d'opérations de renommage sur le nombre opérations de refactoring. Ce qui ne nous donne pas l'importance du refactoring par rapport au projet entier.\\



\subsection{Les gestionnaires de version}

\subsection{``Origin Analysis''}


\subsection{Métriques de procédés et évolution logiciel}


\subsection{Métriques et renommages}


